\documentclass[hidelinks,a4paper,11pt, nofootinbib]{article}
\usepackage{geometry}
\usepackage[spanish, es-tabla]{babel} %es-tabla es para que ponga Tabla en vez de Cuadro en el caption
\usepackage[utf8]{inputenc}
\usepackage[T1]{fontenc}
\usepackage{xspace}
\usepackage{xargs}
\usepackage{fancyhdr}
\usepackage{lastpage}
\usepackage{caratula}
\usepackage[bottom]{footmisc}
\usepackage{amsmath}
\usepackage{amssymb}
\usepackage{algorithm}
\usepackage[noend]{algpseudocode}
\usepackage{array}
\usepackage{xcolor,colortbl}
\usepackage{amsthm}
\usepackage{listings}
\usepackage{soul}

\usepackage{pgf}

\usepackage{graphicx}
\usepackage{sidecap}
\usepackage{amsmath}
\usepackage{wrapfig}
\usepackage{caption}

\setlength{\parindent}{4em}
\setlength{\parskip}{1em}

%Formato de los links
\usepackage{hyperref}
\hypersetup{
  colorlinks   = true, %Colours links instead of ugly boxes
  urlcolor     = blue, %Colour for external hyperlinks
  linkcolor    = blue, %Colour of internal links
  citecolor   = red %Colour of citations
}

\usepackage{comment}
\captionsetup[table]{labelsep=space}


\setlength{\parindent}{4em}
\setlength{\parskip}{0.5em}


%%fancyhdr
\pagestyle{fancy}
\thispagestyle{fancy}
\addtolength{\headheight}{1pt}
\lhead{Ingeniería del Software I+II: TP1}
\rhead{$1º$ cuatrimestre de 2016}
\cfoot{\thepage\ / \pageref{LastPage}}
\renewcommand{\footrulewidth}{0.4pt}
\renewcommand{\labelitemi}{$\bullet$}
\setcounter{section}{-1}


\newcommand{\userstory}[3]{
\begin{tabular}{|r p{10cm}|}
  \hline
  \textbf{como} & #1 \\
  \textbf{quiero} & #2 \\
  \textbf{para} & #3 \\
  \hline
\end{tabular}

}

\newcommand{\critdeacep}[1]{\textbf{Criterio de aceptación:} #1

}
\newcommand{\busvalue}[1]{\textbf{Business Value:} #1

}
\newcommand{\storypoints}[1]{\textbf{Story Points:} #1

}
\newcommand{\primersprint}{\texttt{1er sprint}

}
\newcommand{\segundosprint}{\texttt{2do sprint}

}
\newcommand{\tasks}[1]{\textbf{Tasks:} 

#1}


%%caratula
\materia{Ingeniería del Software I+II}
\titulo{Trabajo Práctico Número 1: User Stories}
%\subtitulo{}
\grupo{Grupo 7}
\integrante{Ciruelos Rodríguez, Gonzalo}{063/14}{gonzalo.ciruelos@gmail.com}
\integrante{Costa, Manuel José Joaquín}{035/14}{manucos94@gmail.com}
\integrante{Gatti, Mathias Nicolás}{477/14}{mathigatti@gmail.com}
\integrante{Maddonni, Axel Ezequiel}{200/14}{axel.maddonni@gmail.com}
\integrante{Thibeault, Gabriel}{114/13}{gabriel.eric.thibeault@gmail.com}

% \fecha{24 de Junio de 2016}
\begin{document}

\maketitle


\section{User Stories}

\subsection*{Buscar bares}
\userstory{usuario de la aplicación}{buscar bares cerca de mí}{poder conocer qué bares cerca de mí tienen wifi y enchufes.}
\primersprint
\critdeacep{Al enviar la posición actual del usuario al sistema, este debe devolver la lista de bares a menos de 400m.}
\busvalue{10}
\storypoints{8}
\tasks{
  \begin{enumerate}
    \item Crear página de búsqueda. (2)
    \item Crear la función de búsqueda, que toma como parámetro la posicin del usuario. (3)
    \item Devolver una lista de jsons para cada bar con su nombre, su distancia, su valoracioón y un link a su página dentro de la aplicación. (1)
    \item[] \textit{Total de horas hombre:} 6
  \end{enumerate}
}

\subsection*{Filtrar búsquedas por feature}
\userstory{usuario de la aplicación}{filtrar los resultados de una búsqueda según el puntaje de las distintas features de los bares}{poder descartar de los resultados los bares que tangan una baja puntuación en la(s) categorías selccionadas}
\segundosprint
\critdeacep{Al seleccionar un filtro, los resultados de la búsqueda ejecutada que aparecen deben pasar las condiciones impuestas por los filtros, y deben ser todos los que las cumplen.}
\busvalue{8}
\storypoints{3}
\tasks{
  \begin{enumerate}
    \item Hacer función que chequee si un bar pasa los criterios especificados. (2)
    \item Filtrar los resultados que se le muestran al usuario utilizando la función dicha antes. (2)
    \item[] \textit{Total de horas hombre:} 4
  \end{enumerate}
}

\subsection*{Agregar bares}
\userstory{moderador}{poder agregar un nuevo bar}{que pueda ser sugerido por la aplicación a los usuarios}
\primersprint
\critdeacep{El moderador debe poder cargar los datos y crear un bar que no existe. El bar debe aparecer en las búsquedas posteriores, reflejando su existencia en el sistema.}
\busvalue{8}
\storypoints{5}
\tasks{
  \begin{enumerate}
    \item Agregar boton en el menu de acciones del moderador para agregar a la base de datos un nuevo bar. (1)
    \item Hacer que al presionar un boton se cargue un formulatio para rellenar con los datos del bar. (3)
    \item Hacer una función que dado un formulario cree una entrada en la base de datos. (2)
    \item[] \textit{Total de horas hombre:} 6
  \end{enumerate}
}


\subsection*{Borrar bares}
\userstory{moderador o dueño del bar}{poder borrar un bar existente}{que no pueda ser sugerido por la aplicación a los usuarios}
\primersprint
\critdeacep{Un bar del catálogo debe poder ser selccionado por el moderador o el dueño del bar para ser eliminado. Si un usuario busca el bar eliminado, este no deberá aparecer. Si un usuario está puntuando o comentando cuando se elimina, recibirá un error. Si dos moderadores intentan borrar simultaneamente el mismo bar, el que lo haga después recibirá un error.}
\busvalue{5}
\storypoints{2}
\tasks{
  \begin{enumerate}
    \item Agregar botón en la página de un bar para que pueda ser borrado por un moderador o su dueño. (2)
    \item Hacer una función que lo elimine de la base de datos. (1)
    \item[] \textit{Total de horas hombre:} 3
  \end{enumerate}
}

\subsection*{Editar bares}
\userstory{moderador o dueño del bar}{poder editar la información de un bar del sistema}{actualizar datos incompletos, incorrectos o desactualizados.
}
\segundosprint
\critdeacep{Dado un bar, el dueño o un moderador debe tener una botón para editarlo dentro de la página del bar. Luego de editarlo, si alguien accede la página del bar, este aparecerá con sus datos actualizados. Si dos moderadores editan un bar al mismo tiempo, la edición definitiva será la del que lo envíe segundo.}
\busvalue{6}
\storypoints{3}
\tasks{
  \begin{enumerate}
    \item Agregar botón en el menú de acciones que tiene el moderador/dueño del bar sobre cada bar para que pueda editar la información. (1)
    \item Hacer que al presionar el botón se cargue un formulario con la inforamción actual del bar, la cual podrá ser editada y actualizada en la base de datos al presionar un botón de confirmación. (4)
    \item[] \textit{Total de horas hombre:} 5
  \end{enumerate}
}
    

\subsection*{Votación}
\userstory{usuario de la aplicación}{poder calificar a un bar con una cantidad de estrellas del $\frac{1}{2}$ (peor) al $5$ (mejor) en todas sus features}{dar una valoración rápida y concreta que ayude al resto de la comunidad en sus futuras búsquedas.}
\segundosprint
\critdeacep{Cuando un bar es calificado por un usuario, el nuevo voto deberá verse reflejado cuando la página del bar sea cargada después de ese momento.}
\busvalue{9}
\storypoints{2}
\tasks{
  \begin{enumerate}
    \item Modificar la representación bar para que pueda almacenar el promedio de votos de cada categoria y la cantida de votos de cada categoria. (2)
    \item Hacer que las funciones que leen la información de un bar lean esa informacion tambien. (2)
    \item[] \textit{Total de horas hombre:} 4
  \end{enumerate}
}

\subsection*{Comentarios}
\userstory{usuario de la aplicación}{escribir comentarios sobre los bares que visito}{compartir detalles que considere importantes para ayudar al resto de la comunidad en futuras búsquedas.}
\segundosprint
\critdeacep{Al escribir y aceptar un nuevo comentario, éste debe aparecer en la vista del bar.}
\busvalue{7}
\storypoints{2}
\tasks{
  \begin{enumerate}
    \item Modificar la representación bar para que pueda almacenar una lista de comentarios. (1)
    \item Modificar la página de vista del bar para que puedan mostrarse los últimos 10 comentarios. (2)
    \item Agregar un botón para que puedan verse todos los comentarios de un bar. (2)
    \item[] \textit{Total de horas hombre:} 5
  \end{enumerate}
}

\subsection*{?`Cómo llegar?}
\userstory{usuario de la aplicación}{conocer la ruta más rápida para llegar al bar deseado desde mi posición actual}{minimizar mi pérdida de tiempo y esfuerzo.}
\segundosprint
\critdeacep{En la vista de un bar deberá haber un botón que me permita ver el camino óptimo al bar desde mi posición. El camino deberá ser efectivamente el camino óptimo.}
\busvalue{9}
\storypoints{3}
\tasks{
  \begin{enumerate}
    \item Agregar un botón en la vista de un bar que permita acceder al mapa de camino más rápido. (1)
    \item Interactuar con la API de Google Maps para obtener el mapa. (3)
    \item[] \textit{Total de horas hombre:} 4
  \end{enumerate}
}

\subsection*{Vista del bar}
\userstory{usuario de la aplicación}{poder acceder a la página de un bar en la aplicación}{ver sus características con mayor detalle}
\primersprint
\critdeacep{La vista del bar debe contener todos los datos actualizados: nombre, foto, y ubicación (y features y comentarios).}
\busvalue{10}
\storypoints{3}
\tasks{
  \begin{enumerate}
    \item Agregar un template para la página correspondiente a la vista de un bar. (3)
    \item Agregar una función para traer los datos actualizados del bar en un json desde la base de datos. (1)
    \item[] \textit{Total de horas hombre:} 4
  \end{enumerate}
}

\subsection*{Filtrar por distancia}
\userstory{usuario de la aplicación}{filtrar los resultados de una búsqueda de bares por máxima distancia}{buscar y distinguir bares según la distancia a la que se encuentran}
\segundosprint
\critdeacep{Al modificar la preferencia de distancia máxima de búsequeda, los bares resultantes de la búsqueda se deberán encontrar a menor o igual distancia que la seleccionada.}
\busvalue{8}
\storypoints{2}
\tasks{
  \begin{enumerate}
    \item Agregar opción de filtro por distancia a la lista de resultados. (1)
    \item Agregar función de filtrado por distancia y filtrar los resultados usandola. (3)
    \item[] \textit{Total de horas hombre:} 4
  \end{enumerate}
}

\subsection*{Volver a página de resultados}
\userstory{usuario de la aplicación}{poder volver a los resultados obtenidos en una búsqueda desde la vista de un bar seleccionado a partir de la misma}{poder acceder a las vistas de otros bares obtenidos en la búsqueda sin tener que volver a realizarla}
\busvalue{5}
\storypoints{3}

\subsection*{Sugerir un nuevo bar}
\userstory{usuario de la aplicación}{sugerir que se agregue un nuevo bar (e indicar potencialmente que soy su dueño)}{que pueda ser aprobado por un mod y de esta forma sea agregado al catálogo (y en caso que lo hubiera indicado, con su debida prueba, sea indicado como dueño del bar).}
\busvalue{7}
\storypoints{5}

\subsection*{Sugerir dueño de un bar}
\userstory{usuario de la aplicación}{proponerme como dueño de un bar que se encuentra en el sistema}{ser reconocido como dueño del bar y poder actualizar su información libremente.}
\busvalue{5}
\storypoints{5}

\subsection*{Aprobar la creación de un bar}
\userstory{moderador}{poder aprobar o rechazar una sugerencia para agregar un nuevo bar (e indicar a un usuario como dueño, si así lo indicó)}{para que la sugerencia sea procesada y removida de la cola.}
\busvalue{5}
\storypoints{5}

\subsection*{Remover comentarios}
\userstory{moderador}{remover comentarios}{aplicar las condiciones de uso de la aplicación.}
\busvalue{6}
\storypoints{1}


\subsection*{Log-in}
\userstory{usuario de la aplicación}{loguearme}{acceder a funcionalidades disponibles sólo a usuarios con cuenta.}
\primersprint
\critdeacep{Al ingresar un usuario y contraseña válidas, el sistema debe iniciar una sesión con ese usuario.}
\busvalue{8}
\storypoints{5}
\tasks{
  \begin{enumerate}
    \item Crear un sistema de log-in. (3)
    \item Crear una página de log-in para que los usuarios puedan ingresar su usuario y contraseña. (2)
    \item[] \textit{Total de horas hombre:} 5
  \end{enumerate}
}

\subsection*{Creación de cuenta}
\userstory{usuario de la aplicación}{crear una cuenta}{poder loguearme y acceder a funcionalidades sólo disponibles a usuarios logueados.}
\segundosprint
\critdeacep{Al ingresar un usuario y contraseña y confirmar la contraseña, si el usuario no existe en el sistema, se creará un nuevo usuario, que podrá loguearse de ahí en adelante.}
\busvalue{9}
\storypoints{3}
\tasks{
  \begin{enumerate}
    \item Hacer una página de creación de usuarios. (1)
    \item Hacer función que cree usuario. (1)
    \item[] \textit{Total de horas hombre:} 2
  \end{enumerate}
}

\subsection*{Crear moderador}
\userstory{moderador}{dar privilegios de moderación a un usuario con cuenta}{permitir que un usuario confiable pueda ser moderador.}
\primersprint
\critdeacep{Si el usario seleccionado existe, se le deberán otorgar privilegios de moderación y tendrá disponibles todas las herramientas de los moderadores a partir de ese momento.}
\busvalue{5}
\storypoints{3}
\tasks{
  \begin{enumerate}
    \item Hacer función que le de privilegios de moderación a una cuenta. (1)
    \item[] \textit{Total de horas hombre:} 1
  \end{enumerate}
}

\end{document}
