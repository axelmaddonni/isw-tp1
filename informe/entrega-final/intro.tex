
Este trabajo se basó en el diseño e implementación de una aplicación que permita a sus usuarios buscar bares cercanos a ellos con Wi-Fi y enchufes.

Cómo modelo de desarrollo de software utilizamos Scrum, que es un proceso iterativo incremental (considerado Agile). En el trabajo se presenta el resultado de los dos primeros sprints, de una duración de alrededor de dos semanas cada uno. Nuestro rol fue tanto de desarrolladores como de Product Owners, por ejemplo porque tomamos las decisiones claves de diseño por nuestra cuenta, y creamos las user stories y los requerimientos nosotros mismos.

En cuanto a la implementación, utilizamos el lenguaje de programación Python. Python es un lenguaje interpretado y dinámicamente tipado. Además, soporta varios paradigmasde programación, entre ellos el orientado a objetos. Esto se va a reflejar en una (a veces significativa) diferencia entre el código y el diseño que veremos más adelante. Esto se debe a que los lenguajes dinámicamente tipados son más flexibles, entonces no es necesario implementar interfaces o clases 100\%
abstractas.

  Como framework para crear la aplicación utilizamos Flask \footnote{\url{http://flask.pocoo.org/}}, que es muy poderoso y muy pequeño a la vez, lo cual nos permitió que no interfiera a la hora de diseñar, y presentar el diseño en diagramas.






